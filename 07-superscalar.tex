\chapter{Parallel and super scalar processors}

Recalling the previously seen formula:
$$T_{exec}=N_{instr}T_{clk} CPI$$
In ideal pipeline organization $CPI$ is equal to 1, however because of hazards
(stalls, wrong branch prediction, etc.) this quantity will never reach one. In
order to reduce the execution time there are two elements on which we can play:
one is $T_{clk}$ (reducing critical path length), the other is $CPI$. Starting
from a MIPS architecture, the choice of which parameter tuning leads to two
different architectures:
\begin{itemize}
  \item Reducing $T_{clk}$ implies having a number of pipeline stages $m$
    greater than 5, leading to \textbf{super-pipelined processor}. We cannot
    exaggerate with $m$ otherwise when the pipeline has to been discarded a lot
    of partially instructions are wasted. Moreover increasing $f_{clk}$ also
    power will increase, so this solution is not very good for low power
    processor.
  \item Having a CPI lower than 1, leading to \textbf{parallel processor}.
    There are two approaches to obtain a parallel process:
    \begin{itemize}
      \item Superscalar processors.
      \item VLIW processors.
    \end{itemize}
\end{itemize}
In last two cases datapath is paralleled, there are several arithmetic units
working in parallel and performing more than one instruction at the same time.
In this case each datapath is still pipelined, so actually pipeline and
parallel approaches are merged together.\\ In \textbf{super-scalar processor}
nothing has to be done at compiler level since it is fully responsibly of
process architecture recognize the parallel instructions, organized them and
execute in parallel. So for the compiler point of view nothing changes because
its output code is generated thinking that it will be executed sequentially.\\
In \textbf{VLIW} (Very Long Instruction Word) instead the processor takes from
the compiler instructions which are already organized in parallel, no need to
extract parallelism or check something. The difference is quite large, the
second approach required much more capability for the compiler.
\newpage

%%%%%%%%%%%%%%%%%%%%%%%%%%%%%%%%%%%%%%%%%%%%%%%%%%%%%%%%%%%%%%%%%%%%%%%%%%%%%%%%
%%%%%%%%%%%%%%%%%%%%%%%%%%%%%%%%%%%%%%%%%%%%%%%%%%%%%%%%%%%%%%%%%%%%%%%%%%%%%%%%
%%%%%%%%%%%%%%%%%%%%%%%%%%%%%%%%%%%%%%%%%%%%%%%%%%%%%%%%%%%%%%%%%%%%%%%%%%%%%%%%
%%%%%%%%%%%%%%%%%%%%%%%%%%%%%%%%%%%%%%%%%%%%%%%%%%%%%%%%%%%%%%%%%%%%%%%%%%%%%%%%
%%%%%%%%%%%%%%%%%%%%%%%%%%%%%%%%%%%%%%%%%%%%%%%%%%%%%%%%%%%%%%%%%%%%%%%%%%%%%%%%
%%%%%%%%%%%%%%%%%%%%%%%%%%%%%%%%%%%%%%%%%%%%%%%%%%%%%%%%%%%%%%%%%%%%%%%%%%%%%%%%
%%%%%%%%%%%%%%%%%%%%%%%%%%%%%%%%%%%%%%%%%%%%%%%%%%%%%%%%%%%%%%%%%%%%%%%%%%%%%%%%
%%%%%%%%%%%%%%%%%%%%%%%%%%%%%%%%%%%%%%%%%%%%%%%%%%%%%%%%%%%%%%%%%%%%%%%%%%%%%%%%
%%%%%%%%%%%%%%%%%%%%%%%%%%%%%%%%%%%%%%%%%%%%%%%%%%%%%%%%%%%%%%%%%%%%%%%%%%%%%%%%
\section{VLIW}
Using VLIW approach, instructions are executed in parallel but the parallelism
is exploited during compilation (static view).
\begin{center}
  \includegraphics[width=0.7\linewidth]{img/img3/6}
\end{center}
It is just needed to take $n$ instructions from the instruction memory and
dispatch them to the functional units, so no branch detection or data
dependencies are required, everything has already been checked. Then it is
possible to specialize the function units, so maybe FU1 is a floating point
unit, FU2 is a divisor, etc. It is required a lot of bandwidth versus
instruction memory, RF and data memory but the hardware is much simpler.
However it is quite common that the compiler is not able to feed at each clock
cycle every functional units (due to data dependencies, the fact that only one
divisor is available, and so on), this implies that a lot of words composing
the fetched  instruction are empty/wasted.
%%%%%%%%%%%%%%%%%%%%%%%%%%%%%%%%%%%%%%%%%%%%%%%%%%%%%%%%%%%%%%%%%%%%%%%%%%%%%%%%
%%%%%%%%%%%%%%%%%%%%%%%%%%%%%%%%%%%%%%%%%%%%%%%%%%%%%%%%%%%%%%%%%%%%%%%%%%%%%%%%
%%%%%%%%%%%%%%%%%%%%%%%%%%%%%%%%%%%%%%%%%%%%%%%%%%%%%%%%%%%%%%%%%%%%%%%%%%%%%%%%
\subsection{Compressing instruction memory}
One possible solution is to decrease the amount of instruction memory required
(since may words will be empty), compressing the information.
\begin{center}
  \includegraphics[width=0.6\linewidth]{img/img3/compr1}
\end{center}
A $U$ means that the FU (Functional Unit) is used, $X$ that is not. Compression
is performed in two steps: in horizontal compression we align to the right all
$X$ (which are nops) and for each row we add a template (i.e. a binary pattern)
to say that in row 1, the first 3 U are referred to FU1, FU2 and FU4 for
instance, then in vertical compression (second stage) empty rows are
canceled.\\ To uncompress: the processor will reads 4 fields at the same time,
if all of them are referring to different FU they can be all dispatched,
instead if some of them referred to the same FU, the processor has to postpone
them.
\begin{center}
  \includegraphics[width=0.6\linewidth]{img/img3/compr2}
\end{center}
After fetching the instructions, processor accesses to the mask: mask0=
10100111 \\ Having 8 functional units:
\begin{verbatim}
mask0(0)=1 ->  W0 in  FU0
mask0(1)=0 ->  NOP in FU1
mask0(2)=1 ->  W2 in FU2
...
\end{verbatim}
Adding mask bits (as many as the number of functional units) it is possible to
obtain a significantly memory saving.

%%%%%%%%%%%%%%%%%%%%%%%%%%%%%%%%%%%%%%%%%%%%%%%%%%%%%%%%%%%%%%%%%%%%%%%%%%%%%%%%
%%%%%%%%%%%%%%%%%%%%%%%%%%%%%%%%%%%%%%%%%%%%%%%%%%%%%%%%%%%%%%%%%%%%%%%%%%%%%%%%
%%%%%%%%%%%%%%%%%%%%%%%%%%%%%%%%%%%%%%%%%%%%%%%%%%%%%%%%%%%%%%%%%%%%%%%%%%%%%%%%
%%%%%%%%%%%%%%%%%%%%%%%%%%%%%%%%%%%%%%%%%%%%%%%%%%%%%%%%%%%%%%%%%%%%%%%%%%%%%%%%
%%%%%%%%%%%%%%%%%%%%%%%%%%%%%%%%%%%%%%%%%%%%%%%%%%%%%%%%%%%%%%%%%%%%%%%%%%%%%%%%
%%%%%%%%%%%%%%%%%%%%%%%%%%%%%%%%%%%%%%%%%%%%%%%%%%%%%%%%%%%%%%%%%%%%%%%%%%%%%%%%
%%%%%%%%%%%%%%%%%%%%%%%%%%%%%%%%%%%%%%%%%%%%%%%%%%%%%%%%%%%%%%%%%%%%%%%%%%%%%%%%
%%%%%%%%%%%%%%%%%%%%%%%%%%%%%%%%%%%%%%%%%%%%%%%%%%%%%%%%%%%%%%%%%%%%%%%%%%%%%%%%
%%%%%%%%%%%%%%%%%%%%%%%%%%%%%%%%%%%%%%%%%%%%%%%%%%%%%%%%%%%%%%%%%%%%%%%%%%%%%%%%
\section{TTA architecture}
Standing for Transport Triggered Architecture, the idea is based on the fact
that every kind of processing can be translated into moves (meaning to take
data from some source and move to a certain destination). Taking as example:
\begin{verbatim}
ADD   R1  R2  R3
\end{verbatim}
In TTA approach, before exploit parallelism we translate the starting
instruction into a sequence of move.
\begin{center}
  \includegraphics[width=0.5\linewidth]{img/img3/7}
\end{center}
Suppose that this new adder has two input registers and provide the output on a
given register, the original operation can be translated as:
\begin{verbatim}
MOVE  o R2
MOVE  T R3
MOVE  R1  R
\end{verbatim}
Where \verb|T| is a special register so that when it is loaded it can trigger
the operation. If all functional units are organized like that (including RF
and data memory) then we can translate all instructions into move operations.\\
The instruction set just consists of one instruction (i.e. move), the compiler
is responsible for translating into moves and organized them in parallel the
instruction sequence. It is possible to exploit pipelining if the operation is
more complicated than a simple addition.
\begin{center}
  \includegraphics[width=0.7\linewidth]{img/img3/8}
\end{center}
In this way the compiler knows the internal latency of each functional unit. It
is also required a proper number of buses to perform all move operations in
parallel.
\begin{center}
  \includegraphics[width=0.7\linewidth]{img/img3/9}
\end{center}
If we have 3 buses in parallel it means that we can perform up to 3 moves in
parallel. Socket is the pattern for each input/output of a functional unit that
determine to which buses this I/O is connected.

%%%%%%%%%%%%%%%%%%%%%%%%%%%%%%%%%%%%%%%%%%%%%%%%%%%%%%%%%%%%%%%%%%%%%%%%%%%%%%%%
%%%%%%%%%%%%%%%%%%%%%%%%%%%%%%%%%%%%%%%%%%%%%%%%%%%%%%%%%%%%%%%%%%%%%%%%%%%%%%%%
%%%%%%%%%%%%%%%%%%%%%%%%%%%%%%%%%%%%%%%%%%%%%%%%%%%%%%%%%%%%%%%%%%%%%%%%%%%%%%%%
\subsection{Software bypassing}
It is like a hardware forwarding, taking as an example:
\begin{verbatim}
ADD R3  R1  R2
ADD R5  R1  R3
\end{verbatim}
And translating them into moves operation:
\begin{verbatim}
MOVE  O   R1
MOVE  T R2
MOVE  R3  R
MOVE  O R1
MOVE  T R3
MOVE  R5  R
\end{verbatim}
Compiler should be able to recognize two things:
\begin{enumerate}
  \item This two instructions have a common operand, so to achieve a faster
    execution we can remove the move command for $R1$ (instruction 4 can be
    canceled).
  \item There is a data dependency since we need the result of first addition
    as input of the second, so we can directly move it to T register.
\end{enumerate}
This leads to:
\begin{verbatim}
MOVE  O   R1
MOVE  T R2
MOVE  T R
MOVE  R5  R
\end{verbatim}
As optional we can put \verb|MOVE R3 R| between 3 and 4 depending on variable
life time. These kind of optimizations are easily performed by compiler.
\begin{verbatim}
\# larger example



















\end{verbatim}

It is required to tune the processor to perform a number of move in parallel:
this amount depends on socket patterns and number of buses (they are
responsible for optimization).

Connecting all i/o ports of a functional units to all buses may lead to a waste
of power: the optimal architecture is the one with the minimum number of dots
and no performance penalty.

This kind of architecture is common in ASIP.

%%%%%%%%%%%%%%%%%%%%%%%%%%%%%%%%%%%%%%%%%%%%%%%%%%%%%%%%%%%%%%%%%%%%%%%%%%%%%%%%
%%%%%%%%%%%%%%%%%%%%%%%%%%%%%%%%%%%%%%%%%%%%%%%%%%%%%%%%%%%%%%%%%%%%%%%%%%%%%%%%
%%%%%%%%%%%%%%%%%%%%%%%%%%%%%%%%%%%%%%%%%%%%%%%%%%%%%%%%%%%%%%%%%%%%%%%%%%%%%%%%
%%%%%%%%%%%%%%%%%%%%%%%%%%%%%%%%%%%%%%%%%%%%%%%%%%%%%%%%%%%%%%%%%%%%%%%%%%%%%%%%
%%%%%%%%%%%%%%%%%%%%%%%%%%%%%%%%%%%%%%%%%%%%%%%%%%%%%%%%%%%%%%%%%%%%%%%%%%%%%%%%
%%%%%%%%%%%%%%%%%%%%%%%%%%%%%%%%%%%%%%%%%%%%%%%%%%%%%%%%%%%%%%%%%%%%%%%%%%%%%%%%
%%%%%%%%%%%%%%%%%%%%%%%%%%%%%%%%%%%%%%%%%%%%%%%%%%%%%%%%%%%%%%%%%%%%%%%%%%%%%%%%
%%%%%%%%%%%%%%%%%%%%%%%%%%%%%%%%%%%%%%%%%%%%%%%%%%%%%%%%%%%%%%%%%%%%%%%%%%%%%%%%
%%%%%%%%%%%%%%%%%%%%%%%%%%%%%%%%%%%%%%%%%%%%%%%%%%%%%%%%%%%%%%%%%%%%%%%%%%%%%%%%

\section{ASIP}
Standing for Application Specific Instruction set Processor, the kind of task
is always the same, like a processor connected to the ABS in a car, or a TV
decoder (more complex but it is always the same). For this processor family it
makes sense optimize the architecture, maintaining the capability to program
it.
To implement asip a common solution is using TTA architecture since it can be
very well optimized.
Depending on the application, we allocate a certain number of functional units,
buses and all possible connections (quite expensive approach):
\begin{verbatim}
\# starting point for the optimization flow


















\end{verbatim}
Then we can simulate it and measure the percentage use of each kind of
resources, based on this analysis the architecture can be simplified by
removing the least used resources.

\# img

Starting from a high implementation cost and a low delay we can remove units up
to the point when performances start to decrease significantly.

Summarizing:

\begin{enumerate}
  \item The initial processor is very similar to a mips, (risk based with 5
    pipeline stage) with a certain ISA. To enhance it, we can provide the ISA
    with a more powerful instruction for a certain application (like fft): this
    leads to a dedicated hardware (i.e. specialized function unit) that perform
    the fft and can be used by a special instruction.

  \item Starting from a given template processor, we can tuned it on our
    requirements (numbers of stages, at which stage allocate the elements we
    want, etc), just limiting to a pipelined organization (no VLWI
    architecture).
\end{enumerate}

%%%%%%%%%%%%%%%%%%%%%%%%%%%%%%%%%%%%%%%%%%%%%%%%%%%%%%%%%%%%%%%%%%%%%%%%%%%%%%%%
%%%%%%%%%%%%%%%%%%%%%%%%%%%%%%%%%%%%%%%%%%%%%%%%%%%%%%%%%%%%%%%%%%%%%%%%%%%%%%%%
%%%%%%%%%%%%%%%%%%%%%%%%%%%%%%%%%%%%%%%%%%%%%%%%%%%%%%%%%%%%%%%%%%%%%%%%%%%%%%%%
%%%%%%%%%%%%%%%%%%%%%%%%%%%%%%%%%%%%%%%%%%%%%%%%%%%%%%%%%%%%%%%%%%%%%%%%%%%%%%%%
%%%%%%%%%%%%%%%%%%%%%%%%%%%%%%%%%%%%%%%%%%%%%%%%%%%%%%%%%%%%%%%%%%%%%%%%%%%%%%%%
%%%%%%%%%%%%%%%%%%%%%%%%%%%%%%%%%%%%%%%%%%%%%%%%%%%%%%%%%%%%%%%%%%%%%%%%%%%%%%%%
%%%%%%%%%%%%%%%%%%%%%%%%%%%%%%%%%%%%%%%%%%%%%%%%%%%%%%%%%%%%%%%%%%%%%%%%%%%%%%%%
%%%%%%%%%%%%%%%%%%%%%%%%%%%%%%%%%%%%%%%%%%%%%%%%%%%%%%%%%%%%%%%%%%%%%%%%%%%%%%%%
%%%%%%%%%%%%%%%%%%%%%%%%%%%%%%%%%%%%%%%%%%%%%%%%%%%%%%%%%%%%%%%%%%%%%%%%%%%%%%%%

\section{Superscalar approach}
\begin{center}
  \includegraphics[width=0.7\linewidth]{img/img3/3}
\end{center}
The \textbf{Decoding and branch prediction} unit hetches and decodes several
(usually 4) instructions at a time from the \textbf{instruction memory},
preparing the data and commands for the execute stage (this operation is done
in several cycles).\\ Then the \textit{dispatched} instructions are sent to the
\textbf{window of execution}.  This is a special kind of buffer/memory where
instructions are collected and ready to be executed. This instructions are not
necessary sorted, but are ruled by constraints (data dependencies,
branches...), hence waiting their execution turn.\\ Then the instruction are
\textit{issued} to a group of \textbf{multistages data path} (different lenght
of stage are supported).\\ Once the result from a datapath is ready they are
\textit{reordered}(in case of out-of-order execution) and \textit{committed}
(similar to the Write Back).\\
This kinds of architectures move most of the scheduling complexity from the
compiler to the harware, in order to increase compatibility.

% We must be able to fetch more than one instruction from the memory, than in
% decoding stage there are more units than in MIPS, the most important one is the
% branch prediction. All instructions to be performed in parallel are stored in
% the so called window of execution, feed to datapaths (more than one stages) and
% finally reach the commit stage. It may happen that two instructions started at
% the same time (like an add and a div) reach the commit stage (where the result
% can be finally stored) in different time, meaning that there is the risk of
% writing the addition result before the division one: this may be a problem due
% to the fact that it may change the semantic of the application. Commit unit has
% to reorganize results, recreate the original order and write them in the order
% intended by the programmer. The key point is that if we allocate more than one
% datapath, these have to be efficiently used, meaning that the processor must be
% able to extract all the potential parallelism


% \subparagraph{Example: window of execution}
% C-code:

% \begin{verbatim}
% for(i=0; i< last; i++) {
%     if(a[i]>a[i+1]) {
%         temp=a[i];
%         a[i]=a[i+1];
%         a[i+1]=temp;
%         change++;
%     }
% }
% \end{verbatim}
% Which is translated into:

% \begin{verbatim}


% \end{verbatim}


% The code is organized in \textbf{basic blocks}, which correspond to a sequence
% of assembler instructions having a single entry-point, only one exit-point and
% no branches inside.\\
% Ideally we could take every block and try to execute instructions inside it in
% parallel, this is simple since there are no branches inside a basic block. If
% we limit our search for parallel execution inside the basic block, we are not
% able to exploit all possibilities of parallel datapaths since inside each basic
% block there are a lot of data dependencies (due to how we write the code). The
% alternative approach consists of having a larger window, so considering more
% instructions coming from different basic blocks the probability to have some of
% them to be performed in parallel will increase. Since instructions now come
% from different basic blocks, there may be branches, so the problem is to
% understand if they can be executed in parallel.\\
% The possibility of performing some instructions coming from different basic
% block depends on the possibility to take a certain branch: we can understand if
% instructions can be performed in parallel only having a branch prediction unit
% telling us if a certain branch will be taken or not, so in this way we can
% understand if we can merge basic block 1+2 or 1+3. \\
% The window of execution and the branch prediction are strictly linked and they
% are both needed to obtain a high efficiency.

%%%%%%%%%%%%%%%%%%%%%%%%%%%%%%%%%%%%%%%%%%%%%%%%%%%%%%%%%%%%%%%%%%%%%%%%%%%%%%%%
%%%%%%%%%%%%%%%%%%%%%%%%%%%%%%%%%%%%%%%%%%%%%%%%%%%%%%%%%%%%%%%%%%%%%%%%%%%%%%%%
%%%%%%%%%%%%%%%%%%%%%%%%%%%%%%%%%%%%%%%%%%%%%%%%%%%%%%%%%%%%%%%%%%%%%%%%%%%%%%%%

\subsection{Reservation stations}
Many processor (Intel) use the following architecture (the one already present
at the beginning of the chapter). In this case the \textit{window of execution}
is responsible to allocate the instruction to corresponding datapath.
\begin{center}
  \includegraphics[width=0.7\linewidth]{img/img3/4}
\end{center}
A different approach consists of using several \textbf{reservation stations} (a
particolar kind of buffer/memory). In this solution each datapath has a
specific queue, as a consequence  each datapath can take instructions only from
it's own queue. The allocation of the instruction to the correct datapath is
performed already during the \textit{dispatch} operation.  The purpose is
always the same: extract all the possible parallelism from the instructions.
\begin{center}
  \includegraphics[width=0.5\linewidth]{img/img3/5}
\end{center}

%%%%%%%%%%%%%%%%%%%%%%%%%%%%%%%%%%%%%%%%%%%%%%%%%%%%%%%%%%%%%%%%%%%%%%%%%%%%%%%%
%%%%%%%%%%%%%%%%%%%%%%%%%%%%%%%%%%%%%%%%%%%%%%%%%%%%%%%%%%%%%%%%%%%%%%%%%%%%%%%%
%%%%%%%%%%%%%%%%%%%%%%%%%%%%%%%%%%%%%%%%%%%%%%%%%%%%%%%%%%%%%%%%%%%%%%%%%%%%%%%%

\subsection{Increasing ILP (Instruction Level Parallelism)}
Usually any assemby code can be divided in what are called \textbf{basic blocks}.
A basic block is a straight-line code sequence with no branches from the entry
to the output, except at the exit.
The code in a basic block has:
\begin{itemize}
  \item One entry point, meaning no code within it is the destination of a jump
    instruction anywhere in the program.
  \item One exit point, meaning only the last instruction can cause the program
    to begin executing code in a different basic block.
\end{itemize}
In general the instruction in a basic block are the one more likely to be
parallelised, hence we would like to have blocks of several instructions.
Usually branch and jump instructions correspond to thje 15/20\% of the total
code, this means that a basic block in usually between 4 to 7 instructions
long (blocks are quite smalls, wich is bad for ILP). Furthermore, the
instruction between a block are usually characterized by a lot of data
dependancies, which further reduces the ILP level.\\ To increase the
instruction level of parallelism, there are many possibilities:
\begin{itemize}
  \item \textit{Branch prediction unit}. The idea is to fill the window of
    execution or the reservations stations understanding in advance the
    instructions that will be executed, so we need a prediction and speculation
    stage, the last one is executing instructions predicted stopping just
    before the final commit (for the final commit we have to wait).
  \item \textit{loop unrolling}: the compiler can increase the number of
    instruction for block unrolling the loop.
  \item \textit{Register renaming}: it allows to eliminate data dependencies.
  \item \textit{Change order of execution}: it allows to eliminate data
    dependencies.
    \begin{itemize}
      \item \textit{In order processor}: simpler, they don't allow any change
        with respect to execution order intended by the programmer.
      \item \textit{Out of order processor}: they allow dynamic change of
        instructions order, either in the back-end either in the front end,
        allowing to exploit better the architecture parallelism.
    \end{itemize}
\end{itemize}

\subsubsection{\Large{Register renaming}}
Let's recall the tree different type of data dependency:
\begin{itemize}
  \item \textbf{RAW} True data dependency / read after write
  \item \textbf{WAR} Anti-dependency / Write after read
  \item \textbf{WAW} Output dependency / Write after write
\end{itemize}
The first one is the only actual data dependancy and the way to solve it are:
stall, rescheduling and forwarding.\\ The other two are just the result of poor
coding, since the dependency is due only to the name of the register. Hence WAR
and WAW may be solved by a practice called \textbf{register renaming}.
Example:
\begin{verbatim}
Original op        ||       Renamed op
R3 <- R1+ R2       ||       R3  <- R1+ R2
R1 <- R4+ R5       ||       R10 <- R4+ R5
\end{verbatim}
This substitution can be performed only if for the following instructions
semantic does not change.

\subsubsection{\Large{Out of order processor}}
In order to understand how a out of order processor works, let's take the
following example:

\begin{verbatim}
1. ADDF R12, R13, R14
2. ADD  R1,  R8,  R9
3. MUL  R4,  R2,  R3
4. MUL  R5,  R6,  R7   |there is a RAW
5. ADD  R10, R5,  R7   |for R5
6. ADD  R11, R2,  R3
\end{verbatim}


Instruction 3 and 4 are potentially in conflict since they employ the same
adder while instruction 5 has a data dependency for R5. We assume that all
instructions have 1 clock cycle latency except \verb|ADDF| (floating point
addition) which has a latency equal to 2.
With term \textit{commit} it is intended the writing of result in the final
destination specified by the programmer, with \textit{completion} the
instruction has been executed, the result has been stored somewhere, maybe in a
temporary register (due to register renaming) but not yet in the final
destination.

\subparagraph{Solution 1: in order for both front-end and back-end}
We assume a degree of parallelism for each stage equal to 2.

\begin{center}
  \begin{tabular}{|c|c|c|c|c|c|}
    \hline
    Cycle    &Fetch     &\multicolumn{3}{|c|}{Execution}  &   Commit  \\
     &       &     F.P.  & ADD & MULT  &               \\ \hline \hline
    1&       $I_1, I_2$  &     -&    -&    -&        -\\
    2&       $I_3, I_4$  &       $I_1$&  $I_2$&  -&        -\\
    3&       $I_5, I_6$  &       $I_1$&  -&    $I_3$ &      -\\
    4&      -&          -&    -&     X &      $I_1,I_2$ \\
    5&      -&          -&    -&    $I_4$&      $I_3$ \\
    6&      -&          -&    $I_5$&    -&      $I_3,I_4$\\
    7&      -&          -&    $I_6$&    -&      $I_5$\\
    8&      -&          -&    -&      -&      $I_6$\\
    \hline
  \end{tabular}
\end{center}
Note that even if $I_3$ is completed and the multiplier is free  we are not
able to issue $I_4$ since we need to commit $I_3$ before beeing able to do
that.
% At the end of cycle 2, $I_2$ has finished, the result is already obtained but
% we cannot perform the completion of $I_2$ since we have to wait for the
% completion of $I_1$ (strict order both in front end both in back end). In
% $4^{th}$ cycle we can complete $I_1$ and $I_2$, not $I_3$ since in the
% completion stage has a degree of parallelism equal to 2.

% We cannot start the multiplication of $I_5$ in clock cycle number 4 since it is
% a in order execution (we have to wait for $I_4$) and due to the data
% dependencies (R5). Assuming to have a forwarding mechanism we can start $I_5$
% just after the end of $I_4$ (we don't have to wait for the completion of
% $I_4$).
\subparagraph{Solution 2: in order issue and out of order execution}
\begin{center}
  \begin{tabular}{|c|c|c|c|c|c|}
    \hline
    Cycle&    Fetch       &\multicolumn{3}{|c|}{Execution}  &   Completion  \\
    &       &     F.P.  & ADD     & MULT  &               \\ \hline \hline
    1&    $I_1, I_2$&       -&    -&    -&      -\\
    2&    $I_3, I_4$&       $I_1$&  $I_2$&  -&      -\\
    3&    $I_5, I_6$&       $I_1$&  -&    $I_3$&      $I_2$\\
    4&      -&          -&    -&    $I_4$&      $I_1,I_3$\\
    5&      -&          -&    $I_5$&    -&      $I_4$\\
    6&      -&          -&    $I_6$&    -&      $I_5$\\
    7&      -&          -&    -&      -&      $I_6$\\
    \hline
  \end{tabular}
\end{center}
Now we can commit $I_2$ before the completion of $I_1$, however for the final
commit we have to wait for $I_1$.

\subparagraph{Solution 3: out of order issue and execution}
\begin{center}
\begin{tabular}{|c|c|c|c|c|c|}
  \hline
  Cycle&    Fetch       &\multicolumn{3}{|c|}{Execution}  &   Completion  \\
  &       &     F.P.  & ADD     & MULT  &               \\ \hline \hline
  1&    $I_1, I_2$&    -    & -   & - &         -     \\
  2&    $I_3,I_4$&        $I_1$&  $I_2$&  -&            -     \\
  3&    $I_5,I_6$&        $I_1$&  -&    $I_3$&          $I_2$   \\
  4&      -&          -&    $I_6$&  $I_4$&          $I_1, I_3$  \\
  5&      -&          -&    $I_5$&  -&            $I_6, I_4$  \\
  6&      -&          -&    -&    -&            $I_5$   \\
  \hline
\end{tabular}
\end{center}
With this solution we are saving one cycle since we can swap the order in the
completion and in the execution stage, issuing $I_6$ before $I_5$.


%%%%%%%%%%%%%%%%%%%%%%%%%%%%%%%%%%%%%%%%%%%%%%%%%%%%%%%%%%%%%%%%%%%%%%%%%%%%%%%%
%%%%%%%%%%%%%%%%%%%%%%%%%%%%%%%%%%%%%%%%%%%%%%%%%%%%%%%%%%%%%%%%%%%%%%%%%%%%%%%%
%%%%%%%%%%%%%%%%%%%%%%%%%%%%%%%%%%%%%%%%%%%%%%%%%%%%%%%%%%%%%%%%%%%%%%%%%%%%%%%%
% \section{Example of Super Scalar Processors}

% The processor we use as an example is a in-order processor called ALPHA21164
% (produced by DEC). Its architecture is based on 4-way (4 datapath in parallel),
% load/store (similar to MIPS for memory access). Block level diagram is the
% following:
% \begin{center}
%   \includegraphics[width=0.7\linewidth]{img/img3/10}
% \end{center}
% The front-end section is pipelined in four stages.
% \begin{itemize}
%   \item $s_0$: access to I-cache to take 4 instructions at the same time (IF).
%   \item $s_1$: validation of read instruction, decode, branch prediction,
%     calculation of the BTS (branch target address).
%   \item $s_2$: slotting/dispatch stage (decide where an instruction has to be
%     allocated/executed).  Check for structural hazards (for instance if all 4
%     operations regard integer operation 2 of them have to been stopped).
%   \item $s_3$: issue stage. Check for data conflict, in particular since it is
%     a in order processor just RAW conflicts may occur, so we verify that all
%     input operands for a certain operation are actually available.
% \end{itemize}
% For the back-end part there is a distintion depending on the kind of
% instruction.\\ For \textbf{integer} instructions there are three pipeline
% stages ($s_4,s_5,s_6$), while for \textbf{F.P.} instructions there are five
% pipeline stages($s_4,s_5,s_6,s_7,s_8$).
% \subparagraph{Load/Store instruction}
% \subparagraph{Branch instruction}


% Looking at this organization it can be noticed that for the front end there are
% 4 pipeline stages while execution units have not the same amount of pipeline
% stages, meaning that for an integer operation (FU1 and FU2) there are three
% stages while for floating point unit (FU3, F4)  5 stages are placed. By
% focusing on these stages we have that front end is made up of $s_0,s_1,s_2, s_3$
% while execution units have $s_4,s_5,s_6$ for integer operations and
% $s_4,...,s_8$ for floating point ones.\\

% \subparagraph{RAW conflict example}
% Let's consider the following example:
% \begin{verbatim}
% I1: R1 <- R2+R3
% I2: R4 <- R1-R5
% I3: R7 <- R8-R9
% I4: F0 <- F2+F4
% \end{verbatim}

% Between I1 and I2 there is a RAW conflict and three instructions regard integer
% data, we can proceed to execute the operations in the following way:

% \begin{center}
%   \begin{tabular}{|c|c|c|c|c|}
%   \hline
%   Cycle&    Stage 0&    Stage 1&      Stage 2&    Stage 3\\
%   \hline
%   1&    I1, I2, I3,I4&    - &         - &         -\\
%   2&    I5,I6,I7,I8&    I1, I2, I3,I4&    -&          -\\
%   3&    ...&        I5,I6,I7,I8&    I1, I2, I3,I4&    -\\
%   4&    ...&        ...&        I5,I6,I7,I8&    I1, I2\\
%   \hline
%   \end{tabular}
% \end{center}

% In clock cycle number 4, since only two integer units are available, we can
% only process I1 and I2, so I3 has to be stopped and I4 cannot be started since
% it is a in order processor (before we have to start I3). In the following step,
% instruction I2 is stopped due to RAW conflict, so although we have 4 parallel
% datapath we have to stall some operations.\\

% Typically Stage 2 and Stage 3 are the so called scoreboarding.

%%%%%%%%%%%%%%%%%%%%%%%%%%%%%%%%%%%%%%%%%%%%%%%%%%%%%%%%%%%%%%%%%%%%%%%%%%%%%%%%
%%%%%%%%%%%%%%%%%%%%%%%%%%%%%%%%%%%%%%%%%%%%%%%%%%%%%%%%%%%%%%%%%%%%%%%%%%%%%%%%
%%%%%%%%%%%%%%%%%%%%%%%%%%%%%%%%%%%%%%%%%%%%%%%%%%%%%%%%%%%%%%%%%%%%%%%%%%%%%%%%

% \subsection{Load-store instruction stages}
% The pipeline stages involved in load or store instructions are stage 4 in which
% address calculation is performed and stage 5 in which access to data-cache
% actually occurs. Two possibilities:

% \begin{itemize}
%   \item \textbf{Hit}: for a reading instruction data is available at the end
%   of Stage 5, for writing the operation is completed in Stage 6.
%   \item \textbf{Miss}: in this case we have to go in L2 cache, so in addition
%   to Stage 5 and Stage 6 we require 6 more stages to support the latency
%   versus L2 cache (up to Stage 12).
% \end{itemize}

%%%%%%%%%%%%%%%%%%%%%%%%%%%%%%%%%%%%%%%%%%%%%%%%%%%%%%%%%%%%%%%%%%%%%%%%%%%%%%%%
%%%%%%%%%%%%%%%%%%%%%%%%%%%%%%%%%%%%%%%%%%%%%%%%%%%%%%%%%%%%%%%%%%%%%%%%%%%%%%%%
%%%%%%%%%%%%%%%%%%%%%%%%%%%%%%%%%%%%%%%%%%%%%%%%%%%%%%%%%%%%%%%%%%%%%%%%%%%%%%%%

% \subsection{Branch}
% In stage 4 the condition for branch is evaluated and in stage 5 branch can be
% resolved, however some other instructions have already been fetched. At this
% point we have to check if the prediction was correct, if it was
% everything is fine, if it was not we have to discharge everything from stage 0
% to stage s4; since 4-instructions parallelism is supported a big penalty
% rises with wrong prediction.


%%%%%%%%%%%%%%%%%%%%%%%%%%%%%%%%%%%%%%%%%%%%%%%%%%%%%%%%%%%%%%%%%%%%%%%%%%%%%%%%
%%%%%%%%%%%%%%%%%%%%%%%%%%%%%%%%%%%%%%%%%%%%%%%%%%%%%%%%%%%%%%%%%%%%%%%%%%%%%%%%
%%%%%%%%%%%%%%%%%%%%%%%%%%%%%%%%%%%%%%%%%%%%%%%%%%%%%%%%%%%%%%%%%%%%%%%%%%%%%%%%
%%%%%%%%%%%%%%%%%%%%%%%%%%%%%%%%%%%%%%%%%%%%%%%%%%%%%%%%%%%%%%%%%%%%%%%%%%%%%%%%
%%%%%%%%%%%%%%%%%%%%%%%%%%%%%%%%%%%%%%%%%%%%%%%%%%%%%%%%%%%%%%%%%%%%%%%%%%%%%%%%
%%%%%%%%%%%%%%%%%%%%%%%%%%%%%%%%%%%%%%%%%%%%%%%%%%%%%%%%%%%%%%%%%%%%%%%%%%%%%%%%
%%%%%%%%%%%%%%%%%%%%%%%%%%%%%%%%%%%%%%%%%%%%%%%%%%%%%%%%%%%%%%%%%%%%%%%%%%%%%%%%
%%%%%%%%%%%%%%%%%%%%%%%%%%%%%%%%%%%%%%%%%%%%%%%%%%%%%%%%%%%%%%%%%%%%%%%%%%%%%%%%
%%%%%%%%%%%%%%%%%%%%%%%%%%%%%%%%%%%%%%%%%%%%%%%%%%%%%%%%%%%%%%%%%%%%%%%%%%%%%%%%
%%%%%%%%%%%%%%%%%%%%%%%%%%%%%%%%%%%%%%%%%%%%%%%%%%%%%%%%%%%%%%%%%%%%%%%%%%%%%%%%
%%%%%%%%%%%%%%%%%%%%%%%%%%%%%%%%%%%%%%%%%%%%%%%%%%%%%%%%%%%%%%%%%%%%%%%%%%%%%%%%
%%%%%%%%%%%%%%%%%%%%%%%%%%%%%%%%%%%%%%%%%%%%%%%%%%%%%%%%%%%%%%%%%%%%%%%%%%%%%%%%
\section{Superscalar out of order processors}
To increase instruction level parallelism out of order processor are exploited.
%%%%%%%%%%%%%%%%%%%%%%%%%%%%%%%%%%%%%%%%%%%%%%%%%%%%%%%%%%%%%%%%%%%%%%%%%%%%%%%%
%%%%%%%%%%%%%%%%%%%%%%%%%%%%%%%%%%%%%%%%%%%%%%%%%%%%%%%%%%%%%%%%%%%%%%%%%%%%%%%%
%%%%%%%%%%%%%%%%%%%%%%%%%%%%%%%%%%%%%%%%%%%%%%%%%%%%%%%%%%%%%%%%%%%%%%%%%%%%%%%%
\subsection {Scoreboarding}
The scoreboard unit is very important in in-order-issue, out-of-order execution.
\begin{center}
  \includegraphics[width=0.7\linewidth]{img/img3/11}
\end{center}
To study the scoreboarding unit we will analyze a processor able to handle all
data conflicts (not only RAW). It is made up of several stages:
\begin{center}
  \begin{tabular}{|m{4.5cm}|m{4cm}|m{4cm}|m{4cm}|}
    \hline
    \textbf{Issue step}& \textbf{Dispatch step} &
    \textbf{Execution} & \textbf{Write step}\\
    \hline
    - Check structural hazard & - Check RAW & & - Check WAR\\
    - Check WAW & & &\\
    \hline
    If check are passed, instruction is issued &
    If check are passed, instruction is dispatched &
    Instruction is executed (time depend on pipe stages) &
    If check are passed, instruction is committed\\
    \hline
  \end{tabular}
\end{center}
If any of these checks is not passed, the instruction has to wait until it
passes, before entering the stage.

\subsubsection{Scoreboard data structures}
Some data structures are allocated in the scoreboarding unit:
\begin{itemize}
\item \textbf{Instruction status table}
\begin{center}
  \begin{tabular}{|l|c|c|c|c|}
    \hline
    Instruction&  Issue&    Read&  Execution&   Write\\
    \hline
    Instr. 1  & & & &  \\
    Instr. 2  & & & &  \\
    Instr. 3  & & & &  \\
    Instr. 4  & & & &  \\
    \hline
  \end{tabular}
\end{center}
For each instruction we have to sign its status for the different scoreboard
steps.

\item \textbf{Functional unit status table}
Each row corresponds to a specific functional unit:
\begin{center}
  \begin{tabular}{|l|c|c|c|c|c|c|c|c|c|}
    \hline
    FU &Busy& Op name &  Dest.reg ($F_i$)& Rs1
    ($F_j$)&
    Rs2 ($F_k$)& $Q_j$& $Q_k$& $R_j$&  $R_k$\\
    \hline
    FU1 & & & & & & & & & \\
    FU2 & & & & & & & & & \\
    FU3 & & & & & & & & & \\
    FU3 & & & & & & & & & \\
    \hline
\end{tabular}
\end{center}
$Q_j$ and $Q_k$ contain the name of the functional units that will generate the
value to be written in $F_j$ and $F_k$ (needed for forwarding)\\ $R_j$ and
$R_k$ are valid data flag that notifies when the data for $F_j$ and $F_k$ are
ready.

\item \textbf{Register result status table}

\begin{verbatim}
  F0   F1   F2   ...   Fn
 ________________...______
|____|____|____|_..._|____|
\end{verbatim}
F0, F1, F2... are registers and for each of them we report which functional
unit will provide the value to be written in that register.
\end{itemize}

% \subparagraph{Summarising}all these operations:
% \begin{center}
%   \begin{tabular}{|l|p{7 cm}|p{7 cm}|}
%     \hline
%     Step &    Condition check &   Actions (if all checks are passed)\\
%     \hline
%     Issue &   \begin{itemize}
%           \item FU free?
%           \item WAW
%           \end{itemize} &     \begin{itemize}
%                       \item Set FU busy.
%                       \item Record $F_i, F_j, F_k, Q_j, Q_k, R_j, R_k$.
%                       \item Store FU name in RF.
%                       \end{itemize} \\

%     Read&   RAW conflict exists if either $R_j$ or $R_k$ is equal to 0.&
%           Send $F_j, F_k$ to the selected FU. \\

%     Execution&- & - \\
%     Write&    WAR conflict($F_i$ for one instruction, for
%     the other instructions already issued
%     $F_j$ and $F_k$).& - \\
%     \hline
%   \end{tabular}
% \end{center}

\subparagraph{Example}
The following set of instructions has to be performed:

\begin{verbatim}
I1:   R4 <- R0  * R2
I2:   R6 <- R4  * R8
I3:   R8 <- R2  + R12
I4:   R4 <- R14 + R16
\end{verbatim}

where I1 and I2 are multiplications. Some conflicts exist:
\begin{enumerate}
  \item I2-I1: RAW.
  \item I2-I3: WAR.
  \item I1-I4: WAW.
  \item All structural hazards.
\end{enumerate}
We assume that addition takes 1 single cycle while multiplication required 5
cycles.\\
\begin{minipage}{\textwidth}
\hspace{-2.5cm} \includegraphics[width=1.5\linewidth]{img/img3/score1}
\end{minipage}
\begin{minipage}{\textwidth}
\hspace{-7cm} \includegraphics[width=1.7\linewidth]{img/img3/score2}
\end{minipage}

% \textit{Cycle 1}: in $Q_j$ and $Q_k$ fields we put "?" since we don't know who
% actually wrote the source registers but we assume that they are already
% available since $R_j=R_k=1$.\\

% \textit{Cycle 2} looking at R6 in the register result status, since this field
% is equal to 0 it means that there are no other instructions in the scoreboard
% who want to write their own result in R6; this implies that no WAW conflict
% occurs for I2.

% The operation of checking WAW is not only a read and compare but it is a search
% in tables, meaning that given the destination register (which is easy to find)
% we have to look at all source registers for each FU to find a match. A
% sequential read operation may take time so the table should be organized as an
% associative memory (at least one part of functional unit states table) in order
% to be able to tell us if a certain word is present or not (although it is a
% much more expensive).

% At cycle 9 we have to flush rows regarding instruction 1 since it is finished.

%%%%%%%%%%%%%%%%%%%%%%%%%%%%%%%%%%%%%%%%%%%%%%%%%%%%%%%%%%%%%%%%%%%%%%%%%%%%%%%%
%%%%%%%%%%%%%%%%%%%%%%%%%%%%%%%%%%%%%%%%%%%%%%%%%%%%%%%%%%%%%%%%%%%%%%%%%%%%%%%%
%%%%%%%%%%%%%%%%%%%%%%%%%%%%%%%%%%%%%%%%%%%%%%%%%%%%%%%%%%%%%%%%%%%%%%%%%%%%%%%%
\subsection{Register renaming}
To understand how register renaming works, let's look at the following list of
instructions:
\begin{verbatim}
i1  R1=R2/R3
i2  R4=R1+R5    RAW with I1
i3  R5=R6+R7    WAR with I2
i4  R1=R8+R9    WAW with I1
\end{verbatim}

By register renaming we can eliminate the last two data dependencies (the first
one cannot be eliminated). In order to apply this technique we have to
distinguish between registers seen by the programmer \textbf{logical registers}
(R1, R2,..., the ones in RF) and \textbf{physical registers} which are not seen
by the programmer. The fist operation is therefore to translate logical
registers into physical register (assuming that physical registers are the ones
multiple of 10):
\begin{verbatim}
i1  R10 = R20/R30
i2  R40 = R10+R50   RAW with I1
i3  R50=R60+R70   WAR with I2
i4  R10=R80+R90   WAW with I1
\end{verbatim}

The first conflict cannot be eliminated, but for the second two it's sufficient
change R50 in I3 into R51 and rename all source operands equal to R50 starting
from I3 into R51.
\begin{verbatim}
i1  R10 = R20/R30
i2  R40 = R10+R50   RAW with I1
i3  R51=R60+R70
i4  R11=R80+R90
\end{verbatim}

To perform register renaming at run time:
There is a list of available physical registers, the access to the first
available location is made using pointer \verb|ptr|, in this way we are able to
perform $R_i \rightarrow R_a$.
Then a rename table is addressed by logical register returning the
corresponding physical register. So regarding destination register we first
have to find an available physical register, increment the pointer (so it will
point at the following location of free list), perform two read operations on
the renaming table for 2 source registers and finally a writing operation on
the same table to save where the destination register is actually stored.
\begin{center}
  \includegraphics[width=1.0\linewidth]{img/img3/26a}
\end{center}
\begin{center}
  \includegraphics[width=1.0\linewidth]{img/img3/26}
\end{center}

However we may have RAW conflicts which cannot be eliminated.
Lookin at the previous example:
\begin{verbatim}
i1  R1=R2/R3
i2  R4=R1+R5    RAW with I1
\end{verbatim}
When we access the renaming table for \verb|r1| (source register in second
operation) the content of thatlocation has not still been updated by the
writing of the first operation. So for the output of the second operation we
have to take either the renaming table, or the free list depending on the
fact that a RAW occurs.
\begin{center}
  \includegraphics[width=0.7\linewidth]{img/img3/27}
\end{center}
Of course it has to been done for all other source registers. As a consequence
even for a simple 2-parallel instruction processor implementation cost of a
renaming unit is not so small.

%%%%%%%%%%%%%%%%%%%%%%%%%%%%%%%%%%%%%%%%%%%%%%%%%%%%%%%%%%%%%%%%%%%%%%%%%%%%%%%%
%%%%%%%%%%%%%%%%%%%%%%%%%%%%%%%%%%%%%%%%%%%%%%%%%%%%%%%%%%%%%%%%%%%%%%%%%%%%%%%%
%%%%%%%%%%%%%%%%%%%%%%%%%%%%%%%%%%%%%%%%%%%%%%%%%%%%%%%%%%%%%%%%%%%%%%%%%%%%%%%%

\subsection{Reordering buffer (ROB)}
In out of order processor execution order may be different from the one
intended by the programmer. In many cases it is allowed, thanks to register
renaming we can delete many conflicts having more freedom in that. However in
the last step, results have to be stored in the exact order intended by the
programmer. By resuming the last example:
\begin{verbatim}
i1  R1 = R2/R3
i2  R4 = R1+R5    RAW with I1
i3  R5 = R6+R7    WAR with I2
i4  R1 = R8+R9    WAW with I1
\end{verbatim}
I1 takes much ore time than I4, if I1 and I4 are issued and executed in
parallel we can expect I4 to be finished much before I1. If at that time we
commit the result of I4 in the logical register, we have changes the semantic
of the program (it's only allowed to write I4 result in the corresponding
physical register). In order to perform this reordering we need:
\begin{center}
  \includegraphics[width=0.4\linewidth]{img/img3/28}
\end{center}
During renaming we insert instructions in a FIFO queue, then we enter the issue
/execution/completion stages which works out of order. After them commit
operation has to be performed in order by following the queue order. Once I1
has been committed it is flushed away from the queue and proceed to I2 commit
(if already available).
\paragraph{Example} (Next 3 pages)
\subparagraph{Issuing} (Next 2 pages) \\
Interested fields are:
\begin{itemize}
\item P1,P2: say if operands source data are available.
\item LD: logical destination register.
\item PRd: physical register associated to LD
\item LPRd= previous physical register associated to LD, it is required so we
  can free the previous physical register.
\end{itemize}
We should take one instruction at the time sequentially, but since our
architecture is 2-way parallel we should take two instructions at the same
time. In the first step we will issue \verb|ld r1, 0(r3)|, then we perform the
issue of second instruction (\# second set of tables), for this second one as
source destination we have to indicate the physical reg for r1 which is P0
(rename table), however it is still not available (p1 is not set). Same thing
occurs for the following instructions. While we are proceeding in issuing, we
are consuming registers from the free list (we almost have no more free regs).
To perform all these things in parallel, it is not just sufficient have
multiple ports for the memories data but we also should be able to perform all
the required tests.
\subparagraph{Execution and commit} (Last Example page)\\
Starting again from the first instruction, the only source operand we have is
already available, so we can execute the first instruction At the time we
commit the first operation, we take the name of previously physical register
associated to r1 which is P8, so we can insert P8 in the free list since the
value previously stored in P8 is no more needed and no more instructions in the
pipeline need that value (it is in order execution so all previous instructions
having r1 as source have already been completed). Proceeding with the following
instructions we also need to update the p1 and p2 fields in the ROB table to
say that P1 is present.
\begin{center}
  \includegraphics[width=1.00\linewidth]{img/img3/rob1}
\end{center}
\begin{center}
  \includegraphics[width=1.00\linewidth]{img/img3/rob2}
\end{center}
\begin{center}
  \includegraphics[width=1.00\linewidth]{img/img3/rob3}
\end{center}
%%%%%%%%%%%%%%%%%%%%%%%%%%%%%%%%%%%%%%%%%%%%%%%%%%%%%%%%%%%%%%%%%%%%%%%%%%%%%%%%
%%%%%%%%%%%%%%%%%%%%%%%%%%%%%%%%%%%%%%%%%%%%%%%%%%%%%%%%%%%%%%%%%%%%%%%%%%%%%%%%
%%%%%%%%%%%%%%%%%%%%%%%%%%%%%%%%%%%%%%%%%%%%%%%%%%%%%%%%%%%%%%%%%%%%%%%%%%%%%%%%
\subsection{Freeing physical registers}
To avoid that free list becomes empty we have to think at a mechanism to put
physical registers in free list.\\ Let's assume R1 to be the logical
destination register and P1 the corresponding physical register, in the
following instructions we will use P1 (since in all source-like occurrences R1
has been renamed P1) but at a certain point R1 will be reused as destination
register: this means that the value written by the first operation in R1 is no
more needed, since all the following operations will used the new value of R1.
This implies that when R1 is again used as a destination we can free P1.\\
However this condition is not sufficient since from the programmer point of
view it works properly, but due to pipelined/out of order, etc. processor there
might be some instructions that still need to use R1 as source register. This
means that a double check has to be performed, both we have to reach a second
R1 assignment both the old value of P1 has became useless since all
instructions requiring it have already been exploited (meaning that all
instructions between two assignments have already reached the read stage so P1
is surely in FUs).\\ This kind of check is quite expensive, a number of
solutions has been proposed:
\begin{enumerate}
  \item For each register a counter is allocated, so starting from
    \verb|R1 <-    OP...| we associate to R1 the physical register P1 and in
    each following instruction where R1 is present as source (like
    \verb|R2 <- OP R1, ..|) R1 is replaced with P1 and the corresponding
    counter for R1 is incremented.  When this instruction reaches read operand
    stage (so the actual P1 value is at the beginning of FU) the counter is
    decremented. At the time we reach a second assignment for R1 (like
    \verb|R1 <- OP R9, ..|) we have to check counter value: if it is equal to 0
    we can free P1, otherwise not since previous value of R1 is still needed
    somewhere. A counter for each physical register is therefore required.
  \item Wait for the second assignment to be committed for freeing P1, in this
    way we are sure that the order intended by the programmer is respected.
\end{enumerate}
These two methods are different, 1) is more efficient in term of time but is
more expensive, 2) is less expensive but required more time.
%%%%%%%%%%%%%%%%%%%%%%%%%%%%%%%%%%%%%%%%%%%%%%%%%%%%%%%%%%%%%%%%%%%%%%%%%%%%%%%%
%%%%%%%%%%%%%%%%%%%%%%%%%%%%%%%%%%%%%%%%%%%%%%%%%%%%%%%%%%%%%%%%%%%%%%%%%%%%%%%%
%%%%%%%%%%%%%%%%%%%%%%%%%%%%%%%%%%%%%%%%%%%%%%%%%%%%%%%%%%%%%%%%%%%%%%%%%%%%%%%%
%%%%%%%%%%%%%%%%%%%%%%%%%%%%%%%%%%%%%%%%%%%%%%%%%%%%%%%%%%%%%%%%%%%%%%%%%%%%%%%%
%%%%%%%%%%%%%%%%%%%%%%%%%%%%%%%%%%%%%%%%%%%%%%%%%%%%%%%%%%%%%%%%%%%%%%%%%%%%%%%%
%%%%%%%%%%%%%%%%%%%%%%%%%%%%%%%%%%%%%%%%%%%%%%%%%%%%%%%%%%%%%%%%%%%%%%%%%%%%%%%%
%%%%%%%%%%%%%%%%%%%%%%%%%%%%%%%%%%%%%%%%%%%%%%%%%%%%%%%%%%%%%%%%%%%%%%%%%%%%%%%%
\section{Memory dependencies}
So far when talking about data dependencies only instructions regarding
registers have been taken into account. However similar problems occur when
handling data from a memory. Let's consider store (st) and load (ld)
operations:
\begin{verbatim}
st  r1, (r2)  // content of register r1 moved in the location pointed by r2
ld r3, (r4)   // read data from location pointed by r4 and put into r1
\end{verbatim}
A data dependency between this two instructions occurs when $r2 = r4$, and this
results in a RAW conflict. In this cases the order of execution cannot be
changed, meaning that if our processor is out of order it has to detect and
manage this situation. This kind of conflict can NOT be detected at compiler
time.
% In addition to that, the execution of instructions is based on
% speculation, especially for out of order processor (just commit stage is
% performed strictly in the order intended by the programmer). \\
\subsubsection{Store commit mechanism}
Recalling that in out-of-order processors there is a difference between
completion and commit of an instruction, the store operation can not be
defenitive. In other words in case of a store the commit of the instruction
would delete the old value from the memory, and in case of out of order
execution it could lead to data hazards.  In order to solve this problem we may
use an additional data structure, called \textbf{store buffer} inserted between
register file and data memory (i.e. data cache).
% Speculation allows to do everything we want except the final write on data
% cache, so we employ the store buffer in the middle.
\begin{center}
  \includegraphics[width=0.7\linewidth]{img/img3/30}
\end{center}
\subsubsection{Load mechanism}
In this way we are able to handle writing operations, but regard reading we
have to take the data from D-cache or store buffer (LOAD op)? Usually these two
works in parallel and both of them are actually cache memories (where each
location has valid, tag and data fields).
\begin{center}
  \includegraphics[width=0.7\linewidth]{img/img3/31}
\end{center}
When performing a load instruction, the ld address is applied to both memories
in parallel, if the data is currently stored in the store buffer we will take
from it, otherwise it means that the data we are looking for will be taken from
data-cache (store buffer has a higher priority with respect to D-cache).
% Store buffer is an additional layer of caching, thanks to it we
% could be allow to change the order of load/store instructions. To handle this
% new scenario some solutions may be exploited:
\subsubsection{Speculation mechanism}
In case we would like the processor to perform out-of-order execution, some
techniques can be exploited.
\begin{itemize}
  \item \textbf{Conservative out of order execution of load}\\
  It is allowed for sure if $r4 \neq r2$ so just by comparing the content of
  these two registers if they are different we can accept the change of order
  between load and store instructions.  Since comparing two registers may be
  expensive, we limit comparison to the LSB part of them (16 LSB instead of
  total 32 bits). If 16 LSB are different, no problem and we can change the
  order, if they are equal we avoid changing the change of order (although it
  may happen that the two MSB parts are different). In this approach the change
  of order is permitted only if we are sure that the two register content is
  different. The comparison is between the current instruction and the
  load/store instruction already issued but not yet completed.
  \item \textbf{Speculative out of order approach}\\
  We guess that $r4 \neq r2$ so load is possibly executed before store. Later
  we check if our initial assumption was correct or not, if yes we proceed
  normally, instead if we discover that $r4=r2$ we have to discharge the two
  instructions and all subsequent instructions in the pipeline. This may result
  in a big penalty, like in the branch.
  \item \textbf{Dynamic prediction}\\
  Same as the speculative approach there is an initial assumption, and later
  the verification. Furthermore in this case there is also a table, pointed by
  the load instruction that we are issuing. The table content are the
  identifiers for all the store operation that cannot be sorted in parallel
  with the current store op. This means that just the first occurence of the
  guess is random, the next ones will be based on the history of the instruction.
  % Like for branch instruction we have a mechanism that dynamically detects data
  % dependencies. At the first occurrence of load and store we make a guess, when
  % we finally know if $r4=r2$ we insert in a binary table the comparison result,
  % so at the next occurrence of these two instructions we may have a better
  % prediction. Just in the first occurrence the guess is random, for the next we
  % have a more confident prediction.
\end{itemize}
%%%%%%%%%%%%%%%%%%%%%%%%%%%%%%%%%%%%%%%%%%%%%%%%%%%%%%%%%%%%%%%%%%%%%%%%%%%%%%%%
%%%%%%%%%%%%%%%%%%%%%%%%%%%%%%%%%%%%%%%%%%%%%%%%%%%%%%%%%%%%%%%%%%%%%%%%%%%%%%%%
%%%%%%%%%%%%%%%%%%%%%%%%%%%%%%%%%%%%%%%%%%%%%%%%%%%%%%%%%%%%%%%%%%%%%%%%%%%%%%%%
%%%%%%%%%%%%%%%%%%%%%%%%%%%%%%%%%%%%%%%%%%%%%%%%%%%%%%%%%%%%%%%%%%%%%%%%%%%%%%%%
%%%%%%%%%%%%%%%%%%%%%%%%%%%%%%%%%%%%%%%%%%%%%%%%%%%%%%%%%%%%%%%%%%%%%%%%%%%%%%%%
%%%%%%%%%%%%%%%%%%%%%%%%%%%%%%%%%%%%%%%%%%%%%%%%%%%%%%%%%%%%%%%%%%%%%%%%%%%%%%%%
%%%%%%%%%%%%%%%%%%%%%%%%%%%%%%%%%%%%%%%%%%%%%%%%%%%%%%%%%%%%%%%%%%%%%%%%%%%%%%%%
%%%%%%%%%%%%%%%%%%%%%%%%%%%%%%%%%%%%%%%%%%%%%%%%%%%%%%%%%%%%%%%%%%%%%%%%%%%%%%%%
%%%%%%%%%%%%%%%%%%%%%%%%%%%%%%%%%%%%%%%%%%%%%%%%%%%%%%%%%%%%%%%%%%%%%%%%%%%%%%%%
\section{Tricks in VLIW architectures}
In this last part some new techniques typical of VLIW processors will be
presented. In this kind of architecture there are some troubles regarding
branches since from the compiler point of view is difficult to know if a
certain branch will be taken or not, so it doesn't know which instructions can
be performed in parallel

\subsection{Branch predication}
It is completely different from branch prediction (which exploits the most
probable direction for a branch), in this new approach instructions coming from
both branches are executed in parallel, there are no mechanisms that try to
guess the branch directions, only at the end, before commit step, we decide
which branch has to be taken (ALAP approach).\\
Both branches are performed in parallel, and when the condition is verified, a
special purpose register (\textit{predicate register}) is set to 1.
At the end, only the op with predicate register set to 1 will be committed.\\
% After instruction 3 there are two possible flows so instructions 4...9 are all
% performed in parallel (usually 4,5,6 have no data dependence with 7,8,9). While
% performing all this instructions they are also associated to a predicate
% register, which is 0 or 1 depending on the fact that a certain instruction
% belongs to left or right flow.\\
Typically a certain instruction is linked to a certain predicate register: $Pi
\leftrightarrow $ instruction i.\\
$Pj Pk = condition$ indicates that the update is always performed, then if the
condition id  true $Pj=1$ and $Pk=0$, otherwhise the opposite.\\
$<Pi> Pj Pk =condition$ means that the update of $P_j$ and $P_k$ is performed
only if $Pi$ is true.
\subparagraph{Example}
\begin{verbatim}
if (a && b)
    j=j+1;
else {
    if(c)
        k++;
    else
        k--;
    m=k*5;
}
i= i+1;
\end{verbatim}
Possible situations:
\begin{center}
  \begin{tabular}{|c|l|}
    \hline
    Condition& Instruction\\
    \hline
    if(a != 0 and b !=0)&         \verb|ADI R2, R2, \#1| (immediate sum)\\
    if not(a != 0 and b !=0) and c!=0&    \verb|ADI R4, R4, \#1|\\
    if not (a != 0 and b !=0) and c=0&    \verb|SBI R4, R4, \#1|\\
    if not (a != 0 and b !=0)&        \verb|MPI R5, R4, \#5|\\
    -&                    \verb|ADI R6, R6, \#1|\\
    \hline
  \end{tabular}
\end{center}
The compiler can identify all this conditions, for each condition a certain
instruction to be performed is associated. The idea is to associate at each of
this condition a predicate register:
\begin{verbatim}
     P1, P2 = EQ (R0, \#0)
<P2> P1, P3 = EQ (R1, \#0)     -> only if P2 is true this writing is performed
<P3>          ADI R2, R2, \#1
<P1> P4, P5 = NEQ(R3, \#0)     (c!=0, only if P1 is correct)
<P4>          ADI R4, R4, \#1  (c!=0 so P4 is true)
<P5>          SBI R4, R4, \#1  (c=0 so P5 is true (and P4 is false))
<P1>          MPI R5, R4, \#5
              ADI R6, R6, \#1
\end{verbatim}
so P3 is true iff (a != 0 and b !=0) so depending on P3 we can execute the
first immediate sum.
% \begin{center}
%   \includegraphics[width=1.1\linewidth]{img/img3/predication}
% \end{center}

% \begin{center}
%   \includegraphics[width=1.1\linewidth]{img/img3/predication2}
% \end{center}
%%%%%%%%%%%%%%%%%%%%%%%%%%%%%%%%%%%%%%%%%%%%%%%%%%%%%%%%%%%%%%%%%%%%%%%%%%%%%%%%
%%%%%%%%%%%%%%%%%%%%%%%%%%%%%%%%%%%%%%%%%%%%%%%%%%%%%%%%%%%%%%%%%%%%%%%%%%%%%%%%
%%%%%%%%%%%%%%%%%%%%%%%%%%%%%%%%%%%%%%%%%%%%%%%%%%%%%%%%%%%%%%%%%%%%%%%%%%%%%%%%
\subsection{Software technique to increase parallelism}
To increase further the instruction level parallelism the compiler can:
\begin{itemize}
  \item \textbf{Loop unrolling}: compiler has the chance to identify loops and
    modify the arrangement of the code with the purpose to increase the degree
    of parallelism.
  \subparagraph{Example}
  \begin{verbatim}
  for(i=0; i< n; i++) {
      add
      add
  }
  \end{verbatim}
  The potential parallelism inside the single operation is only equal to two,
  so if our VLIW have 4 FU so that 4 additions can be performed at the same
  time, using this approach we are exploiting only 50 \% of available
  resources. The loop therefore can be unrolled by halfing the number of
  iterations
  \begin{verbatim}
  for(i=0; i< n/2; i++) {
      add1
      add2
        add3
        add4
  }
  \end{verbatim}
  Providing that there are no data dependencies, in this way we can reach 100
  \% of resources usage.
  \item \textbf{Trace scheduling}:
  \begin{center}
    \includegraphics[width=0.2\linewidth]{img/img3/32}
  \end{center}
  In this example i3 is a branch so by profiling the compiler may know which is
  the more probable path. If this is the case, all instructions coming from the
  most probable path scheduling they are arranged to be performed as much as
  possible in parallel, so we extend the set of instructions to be performed in
  parallel beyond the boundary of branches, obtaining the degree of parallelism
  is better. In this way the execution is out of order and it is determined by
  the compiler. \\
  If the prediction made from the compiler is wrong, we have to discharged the
  wrong operations, so we have to insert some more instructions to be able to
  perform the right operations and make some compensations. For a compiler this
  compensation code is not difficult to be determined (while for dynamic
  approach is much more complex).
\end{itemize}
